\section{The Wide-Field Infrared Survey Explorer Mission}

    The goals of the WISE mission were to observe the entire sky at two near-IR and two mid-IR wavelengths, thus improving and complementing the achievements of the last large all-sky infrared survey by IRAS. In this section, I discuss the details of the WISE mission. Data from the WISE survey constitutes the bulk of my thesis and in the the following section, I will summarize this space based mission, its purpose and the specifications and how its data products can be used to identify circumstellar dust.
   

\subsection{Mission Overview and Mid-Infrared Bands}

\textbf{WISE Satellite Picture from Wright2010\label{fig:wise_sat}}

    The Wide-Field Infrared Survey Explorer mission, (or WISE as it will be henceforth be called), is an Earth orbiting observatory funded by NASA/JPL and launched on December 14th, 2009. WISE is a medium-class explorer mission weighing 750~kg, and orbits at 525~km above the Earth's surface. Details of the entire mission can be found online (\textbf{link site}) or at \citep{Wright2010}. 

   The satellite consists of a 40~cm diameter telescope and four imaging CCDs which are cooled by solid hydrogen crostats. The two near-IR channels image the sky at band centered wavelengths of 3.4\micron and 4.6\micron using HgCdTe arrays, each with 18$\micron$ 1024 $\times$ 1024 pixels. Both of these detectors are cooled to 32~K. The mid-IR channel detectors image the sky at band centered wavelengths of 12\micron and 22\micron and are made from Si:As BIB arrays of the same structure as the near-IR channels. These arrays are cooled to a temperature of 8.2~K. For the remainder of this thesis, I will refer to each of these bands as W1 (3.4\micron), W2 (4.6\micron), W3 (12\micron) and W4 (22\micron). 
   
   The field-of-view (FOV) for each detector is 47' on a side while the pixel size 


\subsection{Review of Data Products}


\subsection{Advantages Over Previous All-Sky Surveys}