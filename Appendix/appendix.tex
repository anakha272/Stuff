


\chapter{Major Debris Disk Studies}

\begin{deluxetable}{lcccl}
\tablecaption{Major Debris Disk Studies}

\tabletypesize{\footnotesize}
\tablewidth{0pt}
\tablecolumns{4}

\tablehead{\colhead{Study } & \colhead{Primary Observing} & \colhead{Sample Age} & \colhead{Sample} & \colhead{Study} \\ 
\colhead{Citation} & \colhead{Instruments} & \colhead{(Myr)} & \colhead{Description} & \colhead{Summary} } 

\startdata
\citet{Spangler2001} & 60\micron, 90\micron ISO & 1--630 & 150 A--K stars & ISO Survey to study evolution of dust \\
\citet{Chen2005} & SM 24\micron, 70\micron & 5--20 & 40 FG Sco Cen stars & fourteen 24\micron\ and seven 70\micron excesses \\
\citet{Rieke2005} & SM 24\micron, IRAS,ISO~25\micron & 5--850 & 266 A Stars & Evolution of A star debris disks \\
\citet{Beichman2006} & SM 24\micron, 70\micron & 150--1500 & 88 FGK Stars & To find good targets for future direct imaging planet searches. 70\micron\ incidence rate of $\sim$13\% \\
\citet{Bryden2006} & SM 70\micron & Median 4000 & 69 FGK Stars & Low frequency of disks for old solar type stars \\
\citet{Chen2006} & SIRS 5.5--35\micron & 10--10000 & 50 B--M stars & Spectra suggest mulitple dust populations \\
\citet{Gorlova2006} & SM 24\micron & 100-120 & 54 B8--K6 & Survey of disks in the Pleiades show 25\% of B--A and 10\% F--K3 stars posssess 24\micron\ excesses \\
\citet{Su2006} & SM 24\micron, 70\micron & 5--850  & 160 A stars & Study of dust evolution around A stars ($\sim$33\%) \\
\citet{Rhee2007} & 12\micron, 25\micron, 100\micron, 160\micron & 5--5000 & 622 B--F stars & Photospheric fits to evaluate excess and study evolution of \iras\ disks \\
\citet{Siegler2007} & SM 24\micron & 50 & 34 B3--M5 & Activity in terrestrial region of FGK stars is common at 50 Myr and decays on timescales of 100~Myr \\
\citet{Trilling2007} & SM 24\micron, 70\micron & 50--10000 & 69 Binary A3--F8 & Incidence of disks around binaries is higher than for singles \\
\citet{Currie2008} & SI 3.6--8\micron, SM 24\micron & 25 & 209 B--K & Survey of disks in NGC 2232. Disk fraction of 25\% at 5 Myr, 50--60\% at 20--25\%. Results suggest most A stars produce icy planets. \\
\citet{Hillenbrand2008} & SM 70\micron & 3--3000 & 328 FGK stars & more than 1/3 of disks may have multi-temperature components, with $\sim$10\% of stars possess 70\micron\ excesses. \\
\citet{Meyer2008} & SM 24\micron & 3--3000 & 309 FGK & 30 stars with excesses. Incidence of 8.5\%--19\% for $<300$~Myr, $<4$\% for older stars. \\
\citet{Rebull2008} & SM 24\micron, 70\micron & 22 & 42 A--K & Study of disks $\beta$~Pictoris moving group. Incidence rates of 23\% for 24\micron\ excesses and $>$37\% for 70\micron\ excesses. \\
\citet{Trilling2008} & SM 24\micron, 70\micron & 190--11000 & 350 A--M stars & Incidence of $\sim$4.2\% for 24\micron, and $\sim16.4$\% 70\micron\ excesses. \\
\citet{Carpenter2009a} & SM 24\micron, 70\micron & 5--17 & 205 B0--M5 stars & 54 disks identified in Upper Sco association. Magnitude of F star 24\micron\ excesses increases from 5--17 Myr with weak confidence. \\
\citet{Bryden2009} & SM 24\micron, 70\micron & 100-12000 & 104 F--M RV planet hosts & Search for disks around RV planets. No large statistical difference between incidence rates of disks around stars with and without planets \\
\citet{Carpenter2009} & SM 24\micron, 70\micron, SI 8.6\micron, SIRS 8--35\micron & 3--3000 & 314 FGK stars & Study of evolution of dust around solar type stars, with 15\% incidence at $<$ 300Myr down to 2.5\% at Gyr ages for 24\micron excesses. Similar decline for 70\micron excess upper envelope \\
\citet{Lawler2009} & SIRS 8.5--12\micron, SIRS 30--34\micron & 100--10000 & 152 GFK stars & 11.8\% incidence rate for 30--34\micron\ excesses (at 100$\times$ Zodiacal Dust), and $<1$\% for 8.5--12\micron\ excesses (at 1000$\times$ Zodiacal Dust).  \\
\citet{Plavchan2009} & SM 24\micron, 70\micron & 8--1100 & 70 A--M stars & 70\micron\ incidence rates of $\sim$4\% around GK stars, and 21\% for FG stars \\
\citet{Koerner2010} & SM 24\micron, 70\micron & \nodata & 634 B-K stars & 24\micron\ excesses at 4.6\% incidence rate and 4.8\% incidence of 70\micron\ excesses. \\
\citet{Chen2011} & SM 24\micron, 70\micron, 3500--10500 \AA Magellan MIKE Spectra & 11--17 & 182 F--G Sco Cen stars & 24\micron\ excess fractions of $\sim$30\%, with 41 new discoveries of PPD and debris disk systems. \\
\citet{Dodson-Robinson2011} & SIRS 32\micron & \nodata & 111 FGKM stars & Survey of mostly planet hosts, show 11 debris disks and show that planets detected by RV searches formed within 240 AU of their stars. \\
\citet{Morales2011} & SIRS 7.5--35\micron, SIRS 5.2--35\micron & $<1$ 1000 & 69 disk stars (A--K) & Observations of stars with SM data. Common warm dust in stars suggest dust not found at same location around all stars. \\
\citet{Chen2012} & SM 24\micron, 70\micron & 11--17 & 215 B--A Sco Cen sars & 51 new discoveries and excess fractions of 24--27\%. \\
\citet{Donaldson2012} & HP 70\micron, 100\micron, 160\micron & 30 & 17 B--M & 6 targets show an excess \\
\citet{Luhman2012} & SM 24\micron, 70\micron, SI 3.6\micron, 4.5\micron, 5.8\micron, 8.0\micron, WISE four bands & 11 & 863 B-M USco Stars & 50 new transitional, evolved and debris disks. $<10$\% of B--G stars show inner primordial disks, and $\sim$25\% at earlier than M5. Disk lifetime longer for lower mass stars. \\
\citet{Urban2012} & SM 24\micron & 670 & 122 A--M stars & Study of disks at LHB ages, with detection of excesses at levels 10\% of photosphere flux at 2--8\% in different SpT bins. \\
\citet{Eiroa2013} & HP 70\micron, 100\micron, 160\micron, HS 250\micron, 350\micron, 500\micron & 100--10000 & 133 FGK stars & DUNES Survey; Disks detected at $f_d$ a several times that of EKB dust at $\sim$20\% incidence rate. A number of disks are resolved. \\
\citet{Chen2014} & SM 24\micron, 70\micron, SIRS 31\micron & 1-10000 & 571 B--K & Analysis of Spitzer/IRS spectra for large disk sample that show double disk tempreatures and spectral features for a number of disks. \\
\citet{Thureau2014} & HP 100\micron, 160\micron & 30--1000 & 86 A stars & DEBRIS Survey; with 24$\pm$5\% \\
\enddata



\end{deluxetable}




\chapter{Something}

\section{The Weighted Excess Metric}
\label{sec:appendix}


We present the full derivation of $\Sigma_{\overline{E[Wj]}}$ for a star at a \WS\ mid-IR band $Wj$, where $j=3\mbox{ or }4$.Starting with Equation~\ref{eq:excess_1}, we arrive at a general form for the weighted excess by adding the individual color excess terms, and multiplying by weights $a_i$

\begin{eqnarray}%\label{eq:gen_weighted_excess}
\overline{E[Wj]} &=&  \sum_{i=1}^{j-1} a_i E[Wi-Wj]\label{eq:gen_weighted_excess}\\
                 &=& \sum_{i=1}^{j-1} a_i\left(Wi-Wj-W_{ij}(B_T-V_T)\right)\label{eq:gen_weighted_excess2}.
\end{eqnarray}

\noindent The weights $a_i$ are normalized and are unknown:

\begin{equation}\label{eq:sum_weights}
\sum_{i=1}^{j-1} a_i \equiv 1. 
\end{equation}

\noindent Our general form for the S/N of the weighted average of the excess at $Wj$ is calculated by dividing equation~\ref{eq:gen_weighted_excess} by the uncertainty in the weighted average, $\sigma_{\overline{E[Wj]}}$. The uncertainty is defined as the quadrature sum of each entry of the Jacobian matrix of $\overline{E[Wj]}$ weighted by its respective uncertainty. The variance of the weighted average is

\begin{equation}\label{eq:wtavgExcessUnc}
      \sigma_{\overline{E[Wj]}}^2 = \sum_{\alpha} \sigma_{\alpha}^2 \left(\frac{\partial \overline{E[Wj]}}{\partial \alpha}\right)^2 + O\left(\sigma_{Wi,Wij} \right) + O(\sigma_{Wi,Wj}),
\end{equation}

\noindent where $\alpha \in \{Wi, Wj, Wij(B_T-V_T)\}$  are terms on the right hand side of Equation~\ref{eq:gen_weighted_excess2}. The cross terms in the Jacobian matrix, $O(\sigma_{Wi,Wij})$ and $O(\sigma_{Wi,Wj})$ are proportional to the covariance of the uncertainties in the \WS\ photometry and the mean \WS\ colors. We ignore the first term, $O(\sigma_{Wi,Wij})$, because $\sigma_{Wij} \sim 0.1 \sigma_{Wi}$ and $W_{ij}$ is only a shallow function of $B_T-V_T$. We also ignore $O(\sigma_{Wi,Wj})$ because the errors on $Wi$ and $Wj$ are not correlated and hence $\sigma_{Wi,Wj}\sim 0$. Thus, Equation~\ref{eq:wtavgExcessUnc} reduces to

\begin{equation}\label{eq:wtavgExcessUnc_reduced}
      \sigma_{\overline{E[Wj]}}^2 \simeq \sum_{\alpha} \sigma_{\alpha}^2 \left(\frac{\partial \overline{E[Wj]}}{\partial \alpha}\right)^2,
\end{equation}

\noindent where $\alpha \in \{Wi, Wj\}$ after removing the photospheric uncertainties from the calculation. We define the significance of the weighted excess at $Wj$ in the same form as in Equation~\ref{eq:combined_significance}:

\begin{equation}\label{eq:combined_sig_appendix}
	    \Sigma_{\overline{E[Wj]}} = \frac{\overline{E[Wj]}}{\sigma_{\overline{E[Wj]}}}. 
\end{equation}

	We proceed with solving for the weights in equation~\ref{eq:gen_weighted_excess}. Using $j=4$ as an example, we can expand equation~\ref{eq:gen_weighted_excess} as 
	
\begin{eqnarray}\label{eq:W4_expanded_1}
\overline{E[W4]} & = & a_1 E[W1-W4] + a_2 E[W2-W4] + a_3 E[W3-W4] \\
				 & = & a_1 (W1-W4-W_{14}) + a_2(W2-W4-W_{24}) + a_3(W3-W4-W_{34}),
\end{eqnarray}

\noindent Inserting $a_3 = 1-a_1-a_2$ into Equation~\ref{eq:W4_expanded_1} produces

\begin{equation}\label{eq:w4_expanded_2}
\overline{E[W4]} =  a_1W1 - a_1W_{14} + a_2 W2 - a_2 W_{24} + W3 - W4 - W_{34} - a_1W3 + a_1 W_{34} - a_2 W3 + a_2 W_{34}. 
\end{equation}

\noindent The variance of $\overline{E[W4]}$ is calculated using Equation~\ref{eq:wtavgExcessUnc_reduced},

\begin{equation}\label{wtd_variance}
\sigma_{\overline{E[W4]}}^2 = a_1^2 \sigma_{W1}^2 + a_2^2 \sigma_{W2}^2 + (1-a_1-a_2)^2\sigma_{W3}^2  + \sigma_{W4}^2.
\end{equation}

\noindent Next we seek solutions for $a_1$ and $a_2$ that minimize the dependence of $\sigma_{\overline{E[W4]}}^2$ on these weights. Thus, by calculating

\begin{equation}\label{eq:mina1}
\left(\frac{\partial \sigma_{\overline{E[W4]}}^2}{\partial a_1}\right) = 0 = 2a_1\sigma_{W1}^2 - 2 \sigma_{W3}^2 + 2a_2\sigma_{W3}^2 + 2a_1\sigma_{W3}^2,
\end{equation}


\begin{equation}\label{eq:mina2}
\left(\frac{\partial \sigma_{\overline{E[W4]}}^2}{\partial a_2}\right) = 0 = 2a_2\sigma_{W2}^2 - 2 \sigma_{W3}^2 + 2a_2\sigma_{W3}^2 + 2a_1\sigma_{W3}^2
\end{equation}


We solve for $a_1$ and $a_2$

\begin{equation}\label{eq:a1solved}
a_1 = \frac{\sigma_{W3}^2\sigma_{W2}^2}{\sigma_{W2}^2\sigma_{W1}^2 + \sigma_{W2}^2\sigma_{W3}^2 + \sigma_{W3}^2\sigma_{W1}^2},
\end{equation}


\begin{equation}\label{eq:a2solved}
a_2 = \frac{\sigma_{W3}^2\sigma_{W1}^2}{\sigma_{W2}^2\sigma_{W1}^2 + \sigma_{W2}^2\sigma_{W3}^2 + \sigma_{W3}^2\sigma_{W1}^2}. 
\end{equation}

\noindent Now, using Equations~\ref{eq:a1solved} and \ref{eq:a2solved}, we recover $a_3$, 

\begin{equation}\label{eq:a3solved}
a_3 = \frac{\sigma_{W2}^2\sigma_{W1}^2}{\sigma_{W2}^2\sigma_{W1}^2 + \sigma_{W2}^2\sigma_{W3}^2 + \sigma_{W3}^2\sigma_{W1}^2}.
\end{equation}

To reduce the form of these weights, we multiply and divide each by $\sigma_{W1}^2\sigma_{W2}^2\sigma_{W3}^2$, to finally obtain the general form presenting us with a general form for each weight 

\begin{equation}\label{eq:ai}
a_i = \frac{1/\sigma_{Wi}^2}{\sum_{i=1}^{j-1}1/\sigma_{Wi}^2}. 
\end{equation}

\noindent  which is valid for either weighted $W3$ ($j=3$) or weighted $W4$ ($j=4$) excesses where $j=3 \mbox{ or } 4$ depending on whether we want to search for $W3$ or $W4$ excesses. We then set $A=\sum_{i=1}^{j-1}1/\sigma_{Wi}^2$, substitute equation \ref{eq:ai} into equation \ref{wtd_variance} to obtain a reduced expression for the variance of the excess ($\sigma_{\overline{E[W4]}}$) and then place that expression into Equation \ref{eq:combined_sig_appendix}. This gives us the final form the for the significance of the weighted excess, which when generalized for for $j=3$ and $j=4$ is

\begin{equation}\label{eq:wtd_significance_appendix2}
\Sigma_{\overline{E[Wj]}} = \frac{\frac{1}{A}\sum\limits_{i=1}^{j-1}\frac{E[Wi-Wj]}{\sigma_i^2}}{\sqrt{\sigma_j^2 + 1/A}},  
\end{equation}

\noindent and is equivalent to the form presented in equation~\ref{eq:combined_significance}.