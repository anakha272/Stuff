\section{Introduction}\label{sec:intro}
    
    %From dust we all come and to dust we all return
    %Make sure this seciton has open questions and then end with questions that I hope to answer
    %discuss that before planets, there was the gas and dust in the system that created them
    %
    Discovery, identification and characterization of circumstellar objects -- objects orbiting other stars -- has gained a foothold in modern astronomy over the last three and a half decades. The detection of the first planets orbiting other stars (i.e., extra-solar, or exoplanets) \textbf{Cite Pulsar exoplanet paper and solar exoplanet paper} has since led to a revolution in our understanding of planet formation. With over 1800 exoplanets (and counting) now confirmed in surprisingly varied configurations, understanding the formation of these systems can advance in greater detail. The hope would then be to determine the frequency and study exo-solar systems like our own by studying the varied architecture of all these other systems. Why not? Ours is the only solar system that has life and is uniquely configured with its 4 inner planets, and 4 outer planets -- in contrast to all others. 
    
    But the planets are not the only aspect which characterize our solar system. The asteroid belt, located between Mars and Jupiter, and the cold Edge-Worth Kuiper belt (EKB) contain thousands, and hundreds of thousands of rocky and ice bodies that either failed to become blah blah
    
    
    These circu
    
    %=============================================================================================
    % IRAS and First Debris Disks
    %=============================================================================================
  
    \subsection{IRAS and Detection of The First Circumstellar Debris Disks }\label{sec:1stdisk_iras}
    
    In 1983, the Infrared Astronomical Satellite (IRAS) was launched through a joint initiative between NASA in the United States, the Netherlands Agency for Aerospace Programmes and the Science and Engineering Research Council in the United Kingdom. By the end of its 10 month mission, IRAS had mapped 96\% of the sky at 12, 25, 60 and 100\micron. This was the first time the entire sky had been imaged in the infrared (IR). 
    
    To calibrate the survey data, observations of certain standard stars with well characterized fluxes were obtained at a greater accuracy than the nominal survey data. Measurements of a few of these standard stars revealed a peculiar behavior leading the calibration team to believe the instrument had malfunctioned. Flux measurements of standard calibration stars like Vega ($\alpha$ Lyr) and other main-sequence stars like Fomalhaut (HD~216956), $\beta$~Pic (HD~39060), and $\epsilon$~Eridani revealed an excess of flux several orders above the predicted photospheric flux at two or more of the longer wavelength bands\citep{Aumann1984}. Figure~\textbf{Place figure from Backman \& Paresce} shows the spectral energy distributions (SED) of these four stars in the Rayleigh-Jeans regime, along with the measured fluxes from IRAS. However, not all the stars observed showed this IR excess emission, as seen in Figure~\textbf{backmannonexcess.png}, indicating a real astrophysical origin. 
    
    Several interpretations on the origin of the IR excess were proposed at the time: perhaps the excess was created by cold blackbody companion to the star, or bipolar jets emanating from the star, or stellar free-free emission. \citet{Aumann1984} and \citet{Backman1993} ruled out these interpretations due to the incompatibility of these models to accurately reproduce the observed flux and spectral shape of the IR excess. For Vega, the only reasonable explanation was the existence of "a shell or ring of relatively large particles distributed around [Vega] at a distance of 85~AU" at an equilibrium temperature of 85~K\citep{Aumann1984}. Similar interpretations were invoked to explain the IR excess around the other IRAS observed stars, with grains in thermal equilibrium around their respective stars heated to temperatures between 50 and 120~K. The physics of this will be explained in \S~\ref{sec:excess_detection}. In other words, for the first time, we had discovered main-sequence stars with circumstellar material, similar to the dust and debris found in our own Solar System.
    
    Additional confirmation arrived in 1984, when Brad Smith and \textbf{Somethign} Terrile were able to resolve the source of the IR excess around $\beta$~Pic. Coronagraphic imaging of the star at 0.89\micron at the 100-inch telescope at Las Campanas revealed an extended, edge-on disk out 400~AU\citep{Smith1984} (see Figure~\textbf{Image of ST beta pic}). This was the first confirmation of an optically thin circumstellar disk around another star. The disk was originally thought to be indicative of on-going planet formation, given the young age of the star. However, as I will discuss in later chapters, this theory would later be revisited and revised in light of the large number of similar circumstellar disk systems discovered over the next couple of decades. 
    
  
        \subsubsection{Origin of  Infrared Excesses and Detecting Circumstellar Material}\label{sec:excess_detection}
            %explain theory of reprocessing of stellar light.         
            When speaking of dust in this thesis, I am mainly referring to particles smaller than a millimeter, though nominally $\sim10--100$\microns in size. These macroscopic particles may be heterogeneously composed of various minerals, silicates, dielectric and refractory particles, as well as volatiles (e.g., water, hydrocarbons). 
            
            In a circumstellar environment, these grains will be basked in stellar radiation. 
            
            
            Grains larger than the nominal wavelength of peak stellar radiation are subject to Mie theory of light scattering (\textbf{do i even need this?}). 
            
            
            The spectral energy distribution (SED) of main-sequence stars with temperatures $T_\star>3000$~K can be approximated to first order as a Planck curve. Emission will peak at wavelengths between $0.4 -- 2.5$\micron, and will behave very close to the Rayleigh-Jeans approximation long-ward of 5\micron. 
            
            
    \subsection{Debris Disk Characteristics - Nomenclature}        
        
    \subsection{IRAS to Spitzer: 30 Years of Debris Disks}\label{sec:30years}
    
        \subsubsection{Cold and Warm Disk Wavelength Regimes}

        \subsubsection{Disk Detection Summary from Major Infrared Observatories}\label{sec:IR_observatories}
        
        \subsubsection{Results from WISE}\label{sec:past_wise}
        
        
    \subsection{Importance of Debris Disks: Signposts for Planetary Systems}
        
        \subsubsection{Dust dissipation processes and Time Scales}
        
        \subsubsection{Collisions to replenish disk}
        
        \subsubsection{Planet-Disk Interactions and evidence of such in the resolved morphology of disks}
    
    
        \subsubsection{Activity in Terrestrial Planet and Habitable Zone}
        
        
        \subsubsection{Notable Examples}

    \subsection{Layout of Dissertation}

        