\section{Introduction}\label{sec:intro}
    
    %From dust we all come and to dust we all return
    %Make sure this seciton has open questions and then end with questions that I hope to answer
    %discuss that before planets, there was the gas and dust in the system that created them
    %
    Discovery, identification and characterization of circumstellar objects -- objects orbiting other stars -- has gained a foothold in modern astronomy over the last three and a half decades. The detection of the first planets orbiting other stars (i.e., extra-solar, or exoplanets) \textbf{Cite Pulsar exoplanet paper and solar exoplanet paper} has since led to a revolution in our understanding of planet formation. With over 1800 exoplanets (and counting) now confirmed in surprisingly varied configurations, understanding the formation of these systems can advance in greater detail. The hope would then be to determine the frequency and study exo-solar systems like our own by studying the varied architecture of all these other systems. Why not? Ours is the only solar system that has life and is uniquely configured with its 4 inner planets, and 4 outer planets -- in contrast to all others. 
    
    But the planets are not the only aspect which characterize our solar system. The asteroid belt, located between Mars and Jupiter, and the cold Edge-Worth Kuiper belt (EKB) contain thousands, and hundreds of thousands of rocky and ice bodies that either failed to become blah blah
    
    
    These circu
    
    %=============================================================================================
    % IRAS and First Debris Disks
    %=============================================================================================
  
    \subsection{IRAS and Detection of The First Circumstellar Debris Disks }\label{sec:1stdisk_iras}
    
    In 1983, the Infrared Astronomical Satellite (IRAS) was launched through a joint initiative between NASA in the United States, the Netherlands Agency for Aerospace Programmes and the Science and Engineering Research Council in the United Kingdom. By the end of its 10 month mission, IRAS had mapped 96\% of the sky at 12, 25, 60 and 100\micron. This was the first time the entire sky had been imaged in the infrared (IR). 
    
    To calibrate the survey data, observations of certain standard stars with well characterized fluxes were obtained at a greater accuracy than the nominal survey data. Measurements of a few of these standard stars revealed a peculiar behavior leading the calibration team to believe the instrument had malfunctioned. Flux measurements of standard calibration stars like Vega ($\alpha$ Lyr) and other main-sequence stars like Fomalhaut (HD~216956), $\beta$~Pic (HD~39060), and $\epsilon$~Eridani revealed an excess of flux several orders above the predicted photospheric flux at two or more of the longer wavelength bands\citep{Aumann1984}. Figure~\textbf{Place figure from Backman \& Paresce} shows the spectral energy distributions (SED) of these four stars in the Rayleigh-Jeans regime, along with the measured fluxes from IRAS. However, not all the stars observed showed this IR excess emission, as seen in Figure~\textbf{backmannonexcess.png}, indicating a real astrophysical origin. 
    
    Several interpretations on the origin of the IR excess were proposed at the time: perhaps the excess was created by cold blackbody companion to the star, or bipolar jets emanating from the star, or stellar free-free emission. \citet{Aumann1984} and \citet{Backman1993} ruled out these interpretations due to the incompatibility of these models to accurately reproduce the observed flux and spectral shape of the IR excess. For Vega, the only reasonable explanation was the existence of "a shell or ring of relatively large particles distributed around [Vega] at a distance of 85~AU" at an equilibrium temperature of 85~K\citep{Aumann1984}. Similar interpretations were invoked to explain the IR excess around the other IRAS observed stars, with grains in thermal equilibrium around their respective stars heated to temperatures between 50 and 120~K. The physics of this will be explained in \S~\ref{sec:excess_detection}. In other words, for the first time, we had discovered main-sequence stars with circumstellar material, similar to the dust and debris found in our own Solar System.
    
    Additional confirmation arrived in 1984, when Brad Smith and \textbf{Somethign} Terrile were able to resolve the source of the IR excess around $\beta$~Pic. Coronagraphic imaging of the star at 0.89\micron at the 100-inch telescope at Las Campanas revealed an extended, edge-on disk out 400~AU\citep{Smith1984} (see Figure~\textbf{Image of ST beta pic}). This was the first confirmation of an optically thin circumstellar disk around another star. The disk was originally thought to be indicative of on-going planet formation, given the young age of the star. However, as I will discuss in later chapters, this theory would later be revisited and revised in light of the large number of similar circumstellar disk systems discovered over the next couple of decades. 
    
  
    \subsection{Detecting Circumstellar Dust}\label{sec:excess_detection}

        \subsection{Dust Grain Characteristics}            
            
        From studies (\textbf{reference to studies}) we know that the dust in a circumstellar dust grains can be composed of macroscopic particles, nominally $<100$\micron in size and heterogeneously composed of various minerals, silicates, other dielectric and refractory particles as well as volatiles compounds (e.g., ice, Carbon Monoxide). 
                    
        Circumstellar dust immersed in the radiation field of its host star, will both scatter and absorb incoming radiation. Though most types of grains are effective scaterers in the optical and near-IR (see \textbf{\S~section on resolved imaging}), a tiny fraction of that light is absorbed and aids in heating the dust,\textbf{need to redo this section}.
        
        However, differentiating scattered versus stellar radiation becomes difficult since main-sequence stars at $T_\star>3000$~K have peak emission in the optical and near-IR ($0.4--2.5$\micron), thus increasing the contrast between scattered and stellar light. Re-emitted thermal emission is much easier to detect, since the peak emission occurs at mid and far-IR wavelengths, regimes with decreased stellar radiation. Therefore, the rest of this section deals with identification of circumstellar dust via its thermal emission.  
        
        
        \subsubsection{Physics of Grain-Radiation Interaction}            
            
            Small dust grains, on the order of tens of microns, are responsible for the the infrared excesses discussed in \S~\ref{sec:1stdisk_iras}. For the remainder of this thesis, I assume that the dust grains are in thermal equilibrium at any given distance with the stellar radiation field. For grains of size $a$ and at a distance $r_D$ from the star, the energy the star receives is conserved with the amount of energy the star radiates. I define the energy absorbed $E_{abs}$ and energy emitted $E_{r}$ by the dust grain to be
            
            \begin{equation}\label{eq:energy_absorbed}
            E_{abs} = \pi \left(\frac{R_\star}{r_D}\right)^2 \int_0^\infty \pi a^2 B(\lambda,T_\star) Q_{abs}(a,\lambda) d\lambda
            \end{equation}
            
            and
            
            \begin{equation}\label{eq:energy_emitted}
            E_{r} = 4\pi \int_0^\infty \pi a^2 B(\lambda,T_D) Q_{abs}(a,\lambda) d\lambda
            \end{equation}
            
            respectively. $Q_{abs}(a,\lambda)$ is the scattering efficiency and can be derived using Mie theory. Conservation of energy yields
            
            \begin{equation}\label{eq:conserve_energy}
             \left(\frac{R_\star}{r_D}\right)^2 \int_0^\infty B(\lambda,T_\star) Q_{abs}(a,\lambda) d\lambda = 4 \int_0^\infty B(\lambda,T_D) Q_{abs}(a,\lambda) d\lambda.
            \end{equation}
            
            Solving for the dust temperature $T_D$ yields an analytic power series, the solution to which can be found in \citet{Backman1993}.  The power series solution can only be derived under the assumption that the absorption and emission coefficient are...
            
            The equilibrium temperature can vary as it depends on the composition and size of the grain. Large blackbody grains that are efficient emitters and absorbers will have equilibrium temperatures according to the following equation 
            
            
            \begin{equation}\label{eq:blackbody_temp}
            T_D = 278 \frac{\left(L_\star/L_\odot \right)^{1/4}}{\sqrt{r_D}}\enspace [K]. 
            \end{equation}
            
        Figure~\textbf{Figure of surface plot} shows how Equation~\ref{eq:blackbody_temp} varies as a function of both stellar luminosity and stellocentric distance. In the majority of cases, grains are are heated to temperatures between 50--300~K. Temperatures above this are rarely found as these grains require close proximity to the star, and are unstable in their orbit. Smaller grains are inefficient emitters and will tend to be hotter than blackbody grains at the same distance from the star. \textbf{maybe put some equation here for it}.
            
           
            
        \subsubsection{Infrared Excesses to Detect Dust}            
            
        The dust temperatures shown in Figure~\textbf{figure of surface plot} indicate that the emission from the grains occurs mainly in the infrared wavelengths and follows to first approximation that of a blackbody $B(\lambda,T_D)$. The emission spectrum of grains that have emissivities which deviate from the blackbody approximation have a steeper curve beyond the peak wavelength of emission $\lambda_0$. In this case the dust SED is 
        
        Emissivities of grains deviating from a blackbody approximation have a 
        
        
        However, to measure any emission from the dust, the amoutn of stellar emission needs to be accuratelry predicted. The majority of dust is detected through unresolved emission using either spectroscopic or photometric imaging techniques. To correctly attribute any observed emission to dust, accurately predicting the stellar emission and subsequent subtraction is key.  Dust at temperatures $T_D<300$~K will have peak emission at $\lambda>9$\micron. Short-ward of that, contrast with the stellar photosphere makes it difficult to 
            
            
            
    \subsection{Debris Disks vs. Primordial Disks}        
        
        Circumstellar disks, depending on the var
        
        
    \subsection{IRAS to Spitzer: 30 Years of Debris Disks}\label{sec:30years}
    
        \subsubsection{Cold and Warm Disk Wavelength Regimes}

        \subsubsection{Disk Detection Summary from Major Infrared Observatories}\label{sec:IR_observatories}
        
        \subsubsection{Results from WISE}\label{sec:past_wise}
        
        
    \subsection{Importance of Debris Disks: Signposts for Planetary Systems}
        
        \subsubsection{Dust dissipation processes and Time Scales}
        
        \subsubsection{Collisions to replenish disk}
        
        \subsubsection{Planet-Disk Interactions and evidence of such in the resolved morphology of disks}
    
    
        \subsubsection{Activity in Terrestrial Planet and Habitable Zone}
        
        
        \subsubsection{Notable Examples}

    \subsection{Layout of Dissertation}

        