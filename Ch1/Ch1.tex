\section{Introduction}\label{sec:intro}
    
    %From dust we all come and to dust we all return
    %Make sure this seciton has open questions and then end with questions that I hope to answer
    %discuss that before planets, there was the gas and dust in the system that created them
    %
    Discovery, identification and characterization of circumstellar objects -- objects orbiting other stars -- has gained a foothold in modern astronomy over the last three and a half decades. The detection of the first planets orbiting other stars (i.e., extra-solar, or exoplanets) \textbf{Cite Pulsar exoplanet paper and solar exoplanet paper} has since led to a revolution in our understanding of planet formation. With over 1800 exoplanets (and counting) now confirmed in surprisingly varied configurations, understanding the formation of these systems can advance in greater detail. The hope would then be to determine the frequency and study exo-solar systems like our own by studying the varied architecture of all these other systems. Why not? Ours is the only solar system that has life and is uniquely configured with its 4 inner planets, and 4 outer planets -- in contrast to all others. 
    
    But the planets are not the only aspect which characterize our solar system. The asteroid belt, located between Mars and Jupiter, and the cold Edge-Worth Kuiper belt (EKB) contain thousands, and hundreds of thousands of rocky and ice bodies that either failed to become blah blah
    
    
    These circu
    
    %=============================================================================================
    % IRAS and First Debris Disks
    %=============================================================================================
    \subsection{IRAS and Detection of The First Circumstellar Debris Disks }\label{sec:1stdisk_iras}
    In 1983, the Infrared Astronomical Satellite (IRAS) was launched through a joint initiative between NASA in the United States, the Netherlands Agency for Aerospace Programmes and the Sciecne and Engineering Research Council in the United Kingdom. By the end of its 10 month mission, IRAS had mapped 96\% of the sky at 12, 25, 60 and 100\micron. This was the first time the entire sky had been imaged in the infrared. 
    
    To calibrate the survey data, observations of certain standard stars with well characterized fluxes were obtained at a greater accuracy than the nominal survey data. Measurements of a few of these standard stars revealed a peculiar behavior leading the calibration team to believe the instrument had malfunctioned. Flux measurements of standard calibration stars like Vega ($\alpha$ Lyr) revealed an excess of flux at two or more of the longer wavelength bands, several orders above the predicted photospheric flux\citep{Aumann1984}. 



                        
        \subsubsection{Using Infrared Excesses to Detect Circumstellar Material}\label{sec:excess_detection}
            %explain theory of reprocessing of stellar light.         
    \subsection{Debris Disk Characteristics - Nomenclature}        
        
    \subsection{IRAS to Spitzer: 30 Years of Debris Disks}\label{sec:30years}
    
        \subsubsection{Cold and Warm Disk Wavelength Regimes}

        \subsubsection{Disk Detection Summary from Major Infrared Observatories}\label{sec:IR_observatories}
        
        \subsubsection{Results from WISE}\label{sec:past_wise}
        
        
    \subsection{Importance of Debris Disks: Signposts for Planetary Systems}
        
        \subsubsection{Dust dissipation processes and Time Scales}
        
        \subsubsection{Collisions to replenish disk}
        
        \subsubsection{Planet-Disk Interactions and evidence of such in the resolved morphology of disks}
    
    
        \subsubsection{Activity in Terrestrial Planet and Habitable Zone}
        
        
        \subsubsection{Notable Examples}

    \subsection{Layout of Dissertation}

        